\chapter{Rules for writing the Thesis}

\section{General Rules}
\begin{itemize}
    \item Code of Conduct: You need to understand and sign the TU Code of Conduct before working on a thesis at TU.
    You can find it at\url{https://www.tuwien.at/fileadmin/Assets/dienstleister/Datenschutz_und_Dokumentenmanagement/Code_of_Conduct_fuer_wissenschaftliches_Arbeiten.pdf} 
    \item Time Planning: Plan your thesis realistically. Check how much time you need for studies and work and other obligations to estimate how much time you can spend per week on your thesis. Especially if you have to learn new things (theoretical knowledge in a new field, a new tool, a new programming language), plan sufficient time for this. Keep in mind that always unforeseen problems can occur. So plan some buffer time.
    \item External Deadlines: Make all deadlines clear before you start the thesis. E.g. if you have any time constraints wrt. projects, visa applications, planned employment or any other time restrictions in your studies, let the supervisor know this before you start working on the thesis. Last minute request will not be accepted. 
    \item 
    \item 
\end{itemize}


\section{Writing the Thesis}

\begin{itemize}
	\item Use the CN group latex Master Thesis template
	\item Continuously document what you are doing
	\begin{itemize}
		\item  Make notes about papers you read
		\item  Document all experiment details.Also if experiments are not successful it is important to document what you did and which errors occurred
		\item Document your software in a way that others can continue to understand and modify/extend the software 
	\end{itemize}
	\item Use US english
	\item Consider to write a paper from your results
	
\end{itemize}


%\subsection{Contents and Structure of the Paper}
%see thesis template



\subsection{Tenses}
\begin{itemize}
	\item Use present tense for state of art
	\item
\end{itemize}
%\ifdraft
\textcolor{magenta}{ \textbf{TZ TODO: add rules and references about tenses}}
%\fi

	
\subsection{References, Copyright and Citations}
\begin{itemize}

	\item citations need to be clearly marked (see code of conduct)
	\item no re-phrasing 
	\item Ideally use no figures copied from somewhere else. If figure are copied, a) the copyright must allow use it and b) they have to be correctly cited
	\item You may use sherpa to identify the copyright rules for  particular Journal. \url{http://www.sherpa.ac.uk/romeo/index.php?la=en&fIDnum=|&mode=simple}
	\item Some useful definitions and rules for plagiarism and self-plagiaraism can be found at \url{https://www.fsdr.at/plagiarism}
	\item Rules how to correctly cite a creative commons figure
see \url{ https://commons.wikimedia.org/wiki/Commons:Reusing_content_outside_Wikimedia}
	\item References: Use books or scientific papers as reference instead of web pages or blog entries
	\item If you have to cite a web page you have to provide the date when you last accessed the page , last accessed at YYYY-MM-DD 
	
\end{itemize}
\textcolor{magenta}{\textbf{TZ TODO: add example for creative commons reference}}

\textcolor{magenta}{\textbf{TZ TODO: add references to code of conduct and plagiarism rules}}


\subsection{Latex Tools}

\textcolor{magenta}{ \textbf{TZ TODO: Add links}}

\section{Tools and Infrastructure}

The following tools are useful:
\begin{itemize}
	\item thesis template
	\item zotero (\url{zotero.org}): Tool  for collecting papers and sharing papers with others (creating a zotero group)
	\item SVN or git for joint paper editing
	\item Overleaf for short term joint editing of latex files
\end{itemize} 

Open Issues
\begin{itemize}
	\item Getting data sets from CN group
	\item Getting access to CN infrastructure (compute cluster, GPU, storage)
	\item Access to NTARC?
	\item provide a template for describing experiments
\end{itemize}

\section{Communication}

The first rule is to stay in contact and inform the supervisor(s) about your progress, questions and difficulties.

So always ask:

\begin{itemize}
	\item If anything is not clear about what you should do
	\item If you do not understand something (e.g., a paper, an equation, a statement)
	\item If you have problems with software, programming, etc.
	\item If you don’t know which papers are relevant and which not
	\item If you have a new idea or want to take a different path.
\end{itemize}

Further rules:
\begin{itemize}
	\item Friday updates: send a brief update to your supervisor(s) every Friday. You can include any ideas, questions or difficulties that you had during the week. If you did not make any progress in the week just send an email saying that you did not make progress.
	\item Use an SVN or git repository to store the latest version of your document
	\item Use meaningful file names: Example: YYYY-MM-DD-YourLastName-DocumentName-version
	\item Send an email to supervisors(s) if a new version to be reviewed is in the SVN
	\item clearly mark all changes in the document that you made compared to the last version. Show how you addressed comments.
\end{itemize}

\section{Reproducibility}




\section{Publishing Papers}

\subsection{Finding suitable Conferences}
\subsubsection{Top Conferences and Journals}

\subsubsection{Conferences and Journal Rankings}


\subsection{Using arxiv}

\section{Open Issues}
\begin{itemize}
	\item put change marking method in template
\end{itemize}

