\chapter{Introduction}

\section{Motivation} \label{sect.motivation}

\todo{With the TODO format you can mark open issues or comments during editing. It will automatically generate a TODO List at the end of the document}


Give a brief overview of the motivation for the thesis.

Provide the Problem Statement: 
\begin{itemize}
	\item Why is your topic an important topic?
	\item Why is it not yet solved?
	\item Why has nobody solved it so far (was it not relevant or not needed so far? was it too difficult so far?)
	\item Why is it not easy to be solved? (why does it need research and a skilled person to solve it?)
	\item Why do you think you can and should solve it now? (e.g., because now it became relevant, or we now have faster computers to make it possible to solve it, or you have a unique idea to solve the problem that no one has tried so far)
	
	With the progressing digitalization of evermore aspects of society, cyber security will always be a relevant issue as no system will ever be fully secure. Preventing possible cyber attacks by developing more robust systems is one way to mitigate the issue, the other is preventing already existing faults from being exploited as not every vulnerability can be patched easily as it is the case with e.g. DoS and bruteforce attacks. To stop such attacks it is necessary to identify them within the vast flow of ordinary network traffic which gives rise to the need of \glspl{ids}. State-of-the-art \glspl{ids} apply two methods to detect occurring attacks: Signature-based detection and statistical anomaly-based detection. Signature-based detection looks for known patterns or signatures within packets and data streams to identify incoming attacks \todo{insert reference to state of the art ids}. Statistical anomaly-based detection focuses on differentiating between normal and abnormal behavior in the system and raises an alert if the latter is identified. The problem with signature-based detection is that unknown attacks are ignored and anomaly-based detection is still not sufficiently accurate and prone to false positives \todo{give examples for IDSs lacking accuracy}. The rise of \gls{ml} gave opportunity to use the mighty pattern recognition capabilities of \glspl{nn} for intrusion detection. As \gls{ml} is a rapidly developing field its steady improvement fueled the advance of \gls{nn} based \glspl{ids} 
	which start to show promising results \todo{give examples for NN based IDSs}. \glspl{nn} however are still mostly trained in a supervised fashion, namely by providing labeled examples of cyber attacks for the \gls{nn} to learn from. This again poses the problem, that only known attacks can be identified, but new attacks that are sufficiently similar to old attacks can also be identified, which is not the case with mere signature-based detection. As with every form of supervised training on \glspl{nn}, labeled data is harder to come by while unlabeled data is often abundant and certainly so for network traffic data. For this reason, self-supervised training/pretraining is seeing increased use in the realm of \gls{ml} \todo{give examples of self supervised machine learning}, as unlabeled data can be used to boost the performance without the need for expensive labeled data. One of the most noteworthy examples of the effectiveness of self-supervised pre-training for Neural Networks in the realm of \gls{nlp} is \gls{bert} \cite{bert} developed by Jacob Devlin \textit{et alteri} from Google AI Language. \gls{bert} is based on the state-of-the-art Transformer architecture \cite{attention} and uses a series of proxy tasks like word masking and next sentence prediction to teach the network about syntax and grammar in a self-supervised fashion. The pre-trained network can then be fine-tuned for more specific tasks like question answering or text classification. Analogous, it would be highly beneficial if these or similar pre-training mechanisms could be used to bolster performance of \gls{ml} based \glspl{ids} by improving the classification of network flows, at the most basic level, into cyber attack vs. no cyber attack. \par
	As the technologies mentioned above are fairly recent (Transformers Dec 2017, \gls{bert} May 2019) and the design space for solutions in the context of \gls{ml} for cyber security is substantial, there has not yet been sufficient inquiry into the possibilities of these new methods when applied to the problems posed by Intrusion Detection and cyber attack classification. \gls{nn} performance also improves with the steadily increasing capabilities of modern \glspl{gpu} which makes this a promising concept which can be improved upon by future more powerful hardware. As I am both versed in the domain of Machine Learning and Cyber Security, i find myself able to contribute to this narrow field of research by writing this thesis.
	
	
\end{itemize}


%\subsection{Objective} \label{sect.objective} 
%TEXT


\section{Research Questions} \label{sect.research_questions}

Here state the exact research questions that you try to answer with the thesis

Good research questions are specific and measurable. For example if you want to show that an anomaly detection method is better than anoter one, do not just say "better" but rather provide details of what you mean by better (e.g., higher speed, lower computational complexity, better detection performance, etc)

\begin{itemize}
	\item R1: 
	\item R2:
	\item R3:
\end{itemize}



\section{Approach} \label{sect.approach}

Here describe briefly what is your approach to solve the problem or to answer the research questions. What methodology did you choose and why (briefly)? e.g., theoretical work, simulations, experiments,...

\section{Contribution} \label{sect.contribution}

Here provide a list of the contributions of your work.

Suggestion (especially for dissertations): provide a table with research questions, methods used to answer each, and major findings and the section in which to find details.

\section{Structure} \label{sect.structure}
Describe the structure of the thesis in 1-2 paragraphs.

In section XXX I provide state of the art...


\section{Support (optional)   } \label{sect.support}
In case your research was supported by a project, you can here mention the project and its objectives

\newpage