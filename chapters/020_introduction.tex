\chapter{Introduction}

% insert info about previous publications below
\begin{Prev.Publ}
	Parts of the contents of this chapter have been published in the following papers:
	\begin{enumerate}
	%\item 
		\item [\lbrack P1\rbrack] \bibentry{cen-cenelec-etsi_smart_grid_coordination_group_smart_2014}In cse you already published parts of your thesis in a paper, please here provide a list of papers in which parts of the content of this chapter already where published  %use the \cite ciations text
	\item [\lbrack P2\rbrack] \bibentry{enisa_smart_2014}
%% 	[\lbrack P1\rbrack] you have to change P1 to whatever number the publication has, manually. sorry.
	\end{enumerate}	
	
	Explanation text, on what parts were adopted from previous publications:\\
	e.g. "The statistical anomaly detection algorithm published in the above mentioned papers and described in this
	Chapter is based on the work done in [xxx]."	
\end{Prev.Publ}




\section{Motivation} \label{sect.motivation}

\todo{With the TODO format you can mark open issues or comments during editing. It will automatically generate a TODO List at the end of the document}


Give a brief overview of the motivation for the thesis.

Provide the Problem Statement: 
\begin{itemize}
	\item Why is your topic an important topic?
	\item Why is it not yet solved?
	\item Why has nobody solved it so far (was it not relevant or not needed so far? was it too difficult so far?)
	\item Why is it not easy to be solved? (why does it need research and a skilled person to solve it?)
	\item Why do you think you can and should solve it now? (e.g., because now it became relevant, or we now have faster computers to make it possible to solve it, or you have a unique idea to solve the problem that no one has tried so far)
\end{itemize}


%\subsection{Objective} \label{sect.objective} 
%TEXT


\section{Research Questions} \label{sect.research_questions}

Here state the exact research questions that you try to answer with the thesis

Good research questions are specific and measurable. For example if you want to show that an anomaly detection method is better than anoter one, do not just say "better" but rather provide details of what you mean by better (e.g., higher speed, lower computational complexity, better detection performance, etc)

\begin{itemize}
	\item R1: 
	\item R2:
	\item R3:
\end{itemize}



\section{Approach} \label{sect.approach}

Here describe briefly what is your approach to solve the problem or to answer the research questions. What methodology did you choose and why (briefly)? e.g., theoretical work, simulations, experiments,...

\section{Contribution} \label{sect.contribution}

Here provide a list of the contributions of your work.

Suggestion (especially for dissertations): provide a table with research questions, methods used to answer each, and major findings and the section in which to find details.

\section{Structure} \label{sect.structure}
Describe the structure of the thesis in 1-2 paragraphs.

In section XXX I provide state of the art...


\section{Support (optional)   } \label{sect.support}
In case your research was supported by a project, you can here mention the project and its objectives

\newpage