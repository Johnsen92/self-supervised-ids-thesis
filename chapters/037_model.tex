\section{Model    \tiny{WHAT will I DO? + Abstraction}  } \label{sec:model}

% insert info about previous publications below
\begin{Prev.Publ}
	Parts of the contents of this chapter have been published in the following papers:
	\begin{enumerate}
		\item [\lbrack P1\rbrack] \bibentry{cen-cenelec-etsi_smart_grid_coordination_group_smart_2014} %use the \cite ciations text
		\item [\lbrack P2\rbrack] \bibentry{enisa_smart_2014}
	\end{enumerate}
	
	Explanation text, on what parts were adopted from previous publications:\\
	e.g. "The statistical anomaly detection algorithm published in the above mentioned papers and described in this
	Chapter is based on the work done in [29]."	
\end{Prev.Publ}

\subsection{Abstraction}
- What will be abstracted?\\
- Where do I compromise?\\


\subsubsection{Architecture} \label{subsec.architecture}
add abstraction of 4 architectures here!



\subsection{Scenarios} \label{subsec.scenarios}


\subsection{Metrics    \tiny{What do i measure?}  }


\subsection{Verification / Validation     \tiny{How do I do that??}  }
Verification: “Are we building the product right”?\\
Verification is a process of evaluating the intermediary work of a software development lifecycle to check if we are in the right track of creating the final product. Checks whether the product is built as per the specified requirement and design specification.\\

Validation: “Are we building the right product”?\\
Validation is the process of evaluating the final product to check whether the software meets the business needs. In simple words the test execution which we do in our day to day life are actually the validation activity which includes smoke testing, functional testing, regression testing, systems testing etc… It determines whether the software is fit for use and satisfy the need.

\newpage