% Copyright (C) 2014-2019 by Thomas Auzinger <thomas@auzinger.name>
%TZ: version slightly modified for ETIT thesis
 
\documentclass[draft,final]{vutinfth} % Remove option 'final' to obtain debug information.

% Load packages to allow in- and output of non-ASCII characters.
\usepackage{lmodern}        % Use an extension of the original Computer Modern font to minimize the use of bitmapped letters.
\usepackage[T1]{fontenc}    % Determines font encoding of the output. Font packages have to be included before this line.
\usepackage[utf8]{inputenc} % Determines encoding of the input. All input files have to use UTF8 encoding.

% Extended LaTeX functionality is enables by including packages with \usepackage{...}.
\usepackage{amsmath}    % Extended typesetting of mathematical expression.
\usepackage{amssymb}    % Provides a multitude of mathematical symbols.
\usepackage{mathtools}  % Further extensions of mathematical typesetting.
\usepackage{microtype}  % Small-scale typographic enhancements.
\usepackage[inline]{enumitem} % User control over the layout of lists (itemize, enumerate, description).
\usepackage{multirow}   % Allows table elements to span several rows.
\usepackage{booktabs}   % Improves the typesettings of tables.
\usepackage{subcaption} % Allows the use of subfigures and enables their referencing.
\usepackage[ruled,linesnumbered,algochapter]{algorithm2e} % Enables the writing of pseudo code.
\usepackage[usenames,dvipsnames,table]{xcolor} % Allows the definition and use of colors. This package has to be included before tikz.
\usepackage{nag}       % Issues warnings when best practices in writing LaTeX documents are violated.
\usepackage{todonotes} % Provides tooltip-like todo notes.
\usepackage{makecell}
\usepackage{float}
\usepackage{adjustbox}
\usepackage{placeins}
\usepackage{svg}
\usepackage{listings}
\usepackage{xcolor}
\usepackage{hyperref}  % Enables cross linking in the electronic document version. This package has to be included second to last.
\usepackage[nopostdot,style=super,nonumberlist,acronym,toc]{glossaries} % Enables the generation of glossaries and lists fo acronyms. This package has to be included last.

\newcolumntype{R}[2]{%
	>{\adjustbox{angle=#1,lap=\width-(#2)}\bgroup}%
	l%
	<{\egroup}%
}
\newcommand*\rot{\multicolumn{1}{R{90}{1em}}}% no optional argument here, please!

% ADDED
%grey boxes for self citation
\usepackage{ntheorem}   % for theorem-like environments
\usepackage{mdframed}   % for framing
\theoremstyle{break}
\theoremheaderfont{\bfseries}
\newmdtheoremenv[
linecolor=gray!20,
leftmargin=00,
rightmargin=00,
backgroundcolor=gray!20,
innertopmargin=5pt,
ntheorem
]{Prev.Publ}{\textbf{Notice of adoption from previous publications in section}}{}
% allow full length citation of sources inside text
\usepackage{bibentry}
\usepackage{natbib}
\nobibliography*

% stop ADDED
\newacronym[plural=IDSs,firstplural=Intrusion Detection Systems (IDS)]{ids}{IDS}{Intrusion Detection System}
\newacronym{ml}{ML}{Machine Learning}
\newacronym[plural=NNs,firstplural=Neural Networks (NN)]{nn}{NN}{Neural Network}
\newacronym{bert}{BERT}{Bidirectional Encoder Representations from Transformers}
\newacronym{nlp}{NLP}{Natural Language Processing}
\newacronym[plural=GPUs,firstplural=Graphics Processing Units (GPU)]{gpu}{GPU}{Graphics Processing Unit}
\newacronym{lstm}{LSTM}{Long Short-Term Memory}
\newacronym{nids}{NIDS}{Network Intrusion Detection System}
\newacronym{iat}{IAT}{Interarrival Time}
\newacronym{bce}{BCE}{Binary Cross Entropy}
\newacronym[plural=RNNs,firstplural=Recurrent Neural Networks (RNN)]{rnn}{RNN}{Recurrent Neural Network}
\newacronym{sgd}{SGD}{Stochastic Gradient Descent}
\makeindex      % Use an optional index.
\makeglossaries % Use an optional glossary.
%\glstocfalse   % Remove the glossaries from the table of contents.

% Define convenience functions to use the author name and the thesis title in the PDF document properties.
\newcommand{\authorname}{Jonas Ferdigg} % The author name without titles.
\newcommand{\thesistitle}{Self-supervised Pre-training on LSTM and  models for Network Intrusion Detection} % The title of the thesis. The English version should be used, if it exists.

% Set PDF document properties
\hypersetup{
    pdfpagelayout   = TwoPageRight,           % How the document is shown in PDF viewers (optional).
    linkbordercolor = {Melon},                % The color of the borders of boxes around crosslinks (optional).
    pdfauthor       = {\authorname},          % The author's name in the document properties (optional).
    pdftitle        = {\thesistitle},         % The document's title in the document properties (optional).
    pdfsubject      = {Subject},              % The document's subject in the document properties (optional).
    pdfkeywords     = {a, list, of, keywords} % The document's keywords in the document properties (optional).
}

\setpnumwidth{2.5em}        % Avoid overfull hboxes in the table of contents (see memoir manual).
\setsecnumdepth{subsection} % Enumerate subsections.

\nonzeroparskip             % Create space between paragraphs (optional).
\setlength{\parindent}{0pt} % Remove paragraph identation (optional).

% Set persons with 4 arguments:
%  {title before name}{name}{title after name}{gender}
%  where both titles are optional (i.e. can be given as empty brackets {}).
\setauthor{}{\authorname}{BSc}{male}
\setadvisor{Univ. Prof. Dipl.-Ing. Dr.-Ing.}{Tanja Zseby}{}{female}

%\setauthor{Pretitle}{\authorname}{Posttitle}{female}
%\setadvisor{Pretitle}{Forename Surname}{Posttitle}{male}



% For bachelor and master theses:
\setfirstassistant{Univ.Ass. Dott.mag.}{Maximilian Bachl}{}{male}

% For dissertations:
\setfirstreviewer{Pretitle}{Forename Surname}{Posttitle}{male}
\setsecondreviewer{Pretitle}{Forename Surname}{Posttitle}{male}

% For dissertations at the PhD School and optionally for dissertations:
\setsecondadvisor{Pretitle}{Forename Surname}{Posttitle}{male} % Comment to remove.

% Required data.
\setregnumber{01226597}
\setdate{01}{01}{2001} % Set date with 3 arguments: {day}{month}{year}.
\settitle{\thesistitle}{Self-Supervised learning for Cyber Security Applications} % Sets English and German version of the title (both can be English or German). If your title contains commas, enclose it with additional curvy brackets (i.e., {{your title}}) or define it as a macro as done with \thesistitle.
%\setsubtitle{Optional Subtitle of the Thesis}{Optionaler Untertitel der Arbeit} % Sets English and German version of the subtitle (both can be English or German).

% Select the thesis type: bachelor / master / doctor / phd-school.
% Bachelor:
%\setthesis{bachelor}
%
% Master:
\setthesis{master}
\setmasterdegree{dipl.} % dipl. / rer.nat. / rer.soc.oec. / master
%
% Doctor:
%\setthesis{doctor}
%\setdoctordegree{rer.soc.oec.}% rer.nat. / techn. / rer.soc.oec.
%
% Doctor at the PhD School
%\setthesis{phd-school} % Deactivate non-English title pages (see below)

% For bachelor and master:
%\setcurriculum{Telecommunications} % Sets
\setcurriculum{Embedded Systems}{Embedded Systems} % Sets the English and German name of the curriculum.
%\setcurriculum{Media Informatics and Visual Computing}{Medieninformatik und Visual Computing} % 

% For dissertations at the PhD School:
\setfirstreviewerdata{Affiliation, Country}
\setsecondreviewerdata{Affiliation, Country}


\begin{document}

\frontmatter % Switches to roman numbering.
% The structure of the thesis has to conform to
%  http://www.informatik.tuwien.ac.at/dekanat

%\addtitlepage{naustrian} % German title page (not for dissertations at the PhD School).
\addtitlepage{english} % English title page.
\addstatementpage

\begin{danksagung*}
\todo{Ihr Text hier.}
\end{danksagung*}

%\begin{acknowledgements*}
%\todo{Enter your text here.}
%\end{acknowledgements*}

\begin{kurzfassung}
\todo{Ihr Text hier.}
\end{kurzfassung}

\begin{abstract}
Machine learning techniques and \glspl{dnn} have found their way into various disciplines and their possible 
benefits are explored for a diverse range of application. The pattern matching capabilities of modern day machine learning models 
have long surpassed expert systems or even humans in narrow applications. Their ability to accurately classify seemingly complex data
makes them well suited to also be used in the context of \gls{nid}. While supervised learning is still most effective when
training machine learning models, its feasibility is often stifled by a lack of expensive labeled data. For this and
other reasons researchers at the forefront of machine learning development, especially in the field of \gls{nlp}, have began 
to pre-train their models on large amounts of unlabeled data to overcome the scarcity of labeled data. A commonly used pattern e.g. used to train 
Google's \gls{bert} model, is to pre-train large scale machine learning models in a self-supervised manner. This is done by tasking the model to either reconstruct omitted parts of information from the input data, predicting future input or asking other questions about the input data
to which the answer is derivable from the unlabeled data. Only a small amount of labeled data is then used to fine-tune the model to perform 
the target downstream task. Inspired by the achievements of models like \gls{bert} and its successors we used the same methods to increase classification accuracy for deep learning based \gls{nids}.
In our research we try to answer the question whether pre-training paradigms used in \gls{nlp} can improve classification accuracy for deep learning based \gls{nids}. We performed pre-training on \gls{lstm} and transformer encoder models with a set of devised auto encoding and auto regression based self-supervised training methods to improve binary classification of network traffic records. After pre-training we use supervised fine-tuning with a small amount of labeled data to teach the model how to classify the data into attack and benign flows. As training data we used a flow representations of the CIC-IDS2017 and UNSW-NB15 \gls{nid} datasets with the flow key <dstIp, srcIp, dstPort, srcPort, protocolId>. Our flows consist of a sequence of tensors containing packet and flow specific features. Our results show that classification accuracy can be improved through pre-training, but only in specific instances. Further inquiry is needed to see if our results can be generalized. 
\end{abstract}

% Select the language of the thesis, e.g., english or naustrian.
\selectlanguage{english}

% Add a table of contents (toc).
\tableofcontents % Starred version, i.e., \tableofcontents*, removes the self-entry.

% Switch to arabic numbering and start the enumeration of chapters in the table of content.
\mainmatter

%\input{intro.tex} % A short introduction to LaTeX.
%\input{chapter/}

\chapter{Introduction}

% insert info about previous publications below
\begin{Prev.Publ}
	Parts of the contents of this chapter have been published in the following papers:
	\begin{enumerate}
	%\item 
		\item [\lbrack P1\rbrack] \bibentry{cen-cenelec-etsi_smart_grid_coordination_group_smart_2014}In cse you already published parts of your thesis in a paper, please here provide a list of papers in which parts of the content of this chapter already where published  %use the \cite ciations text
	\item [\lbrack P2\rbrack] \bibentry{enisa_smart_2014}
%% 	[\lbrack P1\rbrack] you have to change P1 to whatever number the publication has, manually. sorry.
	\end{enumerate}	
	
	Explanation text, on what parts were adopted from previous publications:\\
	e.g. "The statistical anomaly detection algorithm published in the above mentioned papers and described in this
	Chapter is based on the work done in [xxx]."	
\end{Prev.Publ}




\section{Motivation} \label{sect.motivation}

\todo{With the TODO format you can mark open issues or comments during editing. It will automatically generate a TODO List at the end of the document}


Give a brief overview of the motivation for the thesis.

Provide the Problem Statement: 
\begin{itemize}
	\item Why is your topic an important topic?
	\item Why is it not yet solved?
	\item Why has nobody solved it so far (was it not relevant or not needed so far? was it too difficult so far?)
	\item Why is it not easy to be solved? (why does it need research and a skilled person to solve it?)
	\item Why do you think you can and should solve it now? (e.g., because now it became relevant, or we now have faster computers to make it possible to solve it, or you have a unique idea to solve the problem that no one has tried so far)
\end{itemize}


%\subsection{Objective} \label{sect.objective} 
%TEXT


\section{Research Questions} \label{sect.research_questions}

Here state the exact research questions that you try to answer with the thesis

Good research questions are specific and measurable. For example if you want to show that an anomaly detection method is better than anoter one, do not just say "better" but rather provide details of what you mean by better (e.g., higher speed, lower computational complexity, better detection performance, etc)

\begin{itemize}
	\item R1: 
	\item R2:
	\item R3:
\end{itemize}



\section{Approach} \label{sect.approach}

Here describe briefly what is your approach to solve the problem or to answer the research questions. What methodology did you choose and why (briefly)? e.g., theoretical work, simulations, experiments,...

\section{Contribution} \label{sect.contribution}

Here provide a list of the contributions of your work.

Suggestion (especially for dissertations): provide a table with research questions, methods used to answer each, and major findings and the section in which to find details.

\section{Structure} \label{sect.structure}
Describe the structure of the thesis in 1-2 paragraphs.

In section XXX I provide state of the art...


\section{Support (optional)   } \label{sect.support}
In case your research was supported by a project, you can here mention the project and its objectives

\newpage
\chapter{Background} \label{sec:background}

\glspl{ann} have shown great improvements over the last years due to increasing computational power, more sophisticated models and smarter training algorithms \cite{reboot_acgan}, \cite{unsupervised_learning_video_segmentation}. \gls{ml} and \glspl{ann} have long found their way into many commercial applications and many scientific fields have successfully applied this relatively new method of data processing to further their own research \cite{alpha_fold}, \cite{ai_medicine}, \cite{ai_antibiotic}. It was only logical that researchers and companies have also started to look into the possible benefits this emerging technology could have for Network Security applications \cite{kitsune}, \cite{ml_ids_survey}. \glspl{ann} are especially suited for \glspl{ids} due to their capability to classify data with high accuracy based on seemingly complex patterns. To harness the power of \gls{ml} for the purpose of Network Security, we made use of existing methods and models, which we will summarize in this section.

\section{Notation} \label{sec:background:terminology:notation}

Throughout the thesis, sequences are denoted $x^{(n)}$ while an element in a vector or matrix would be denoted in subscript e.g. $x_i$ or $W_{i,j}$ respectively, with matrices always written in capital letters. Superscript letters (except for $T$, which is the \textit{transpose} operator and will never be used as label) without parenthesis are part of the variable name, e.g. $W^Q$ is the \textit{query} matrix in the attention function. Deviations from this notation are stated explicitly.

\section{Machine Learning} \label{sec:background:ml}

Machine learning describes the study of computer algorithms which are \textit{trained} or \text{fitted} to optimize a given criterion without the need to specifically program them. The algorithm constructs a model based on input-output pairs which describe how the model should behave.
To ensure that the algorithm is learning patterns and structures of problems and not only memorizing input-output data pairs, the training process is often split into two phases: \textit{Training} phase and \textit{validation} phase. In the training phase, the algorithm processes the data and produces a best guess for the output. In the validation phase, the network processes unseen data, i.e. data not used during training, to ensure that the model did not just learn the training data by heart. 
Popular traditional machine learning algorithms and models used for classification are \gls{knn}, \gls{svm}, Naive Bayes or \gls{dtc}
\par

\section{Artificial Neural Networks} \label{sec:background:ann}

\glspl{ann} are a type of Machine Learning algorithm used for classification and prediction. Named after their resemblance to neurons in a brain, \glspl{ann} are systems comprised of connected nodes called \textit{artificial neurons}. Analogous to synapses, nodes communicate \textit{via} connections called \textit{edges} by sending "signals" to other nodes. Signals are represented as scalar real numbers. The output signal from a sending node is multiplied by the weight of the edge the signal is "traveling" on. Each node calculates its output signal by applying a non-linear function to the sum of its input signals. Signals travel forward through the network from the first to the last layer, but usually not within layers. The resulting computations can be summarized as a combination of function compositions and matrix multiplications $g(x) := f^L(W^Lf^{L-1}(W^{L-1}...f^1(W^1x)...))$ where $L$ is the number of layers, $W^l, l \in \{1,...,L\}$ the weights connecting nodes of the prior layer to layer $l$ and $f^l$ the activation function of the layer. $W^l$ can also be written as series $(w^l_{jk})$ where $w^l_{jk}$ is the weight between the $k$-th node in layer $l-1$ and the $j$-th node in layer $l$. \par There are various types of \glspl{ann} like \glspl{rnn} or \glspl{cnn} which have many derivations themselves but they all operate on the before stated principal of signals traveling through the network, which get transformed at each node by a differentiable non-linear function. The most popular non-linear function at this time is the \gls{relu} function. Without training, an \gls{ann} performs an input transformation that depends on the initialization values of its weights, often called \textit{parameters}. \par

%By processing the input data and comparing it with the desired output data the algorithm adjusts its internal parameters i.e. its weights through the process of \gls{sgd} to produce better results in the next iteration. 
%The influence on the output of every weight in the model is then calculated with \textit{backpropagation} \ref{sec:background:backprop} and the weights of the model are changed in the direction of the gradient. \par 

%Commonly used loss functions for regression tasks are the \gls{mae} or \gls{mse} loss or the very popular Cross-entropy loss for classification.

During training, a metric of difference, often called \textit{loss}, is then calculated between the output of the model and the ground truth, i.e. the expected output. The function used to calculate the loss is called a \textit{loss function} or \textit{cost function} and must be differentiable. The network is trained to perform a desired transformation by adjusting its weights/parameters through virtue of \textit{backpropagation} and \gls{sgd}. The network produces output $\hat{y}$ at the last layer after processing input $x$. A scalar cost/loss value is calculated by the loss function $C(\hat{y}, y)$ as a measure of difference between the networks output $\hat{y}$ and the target output $y$. For classification tasks, the loss function is usually \gls{cel}, and for regression \gls{sel} or L1 loss is typically used. Backpropagation \ref{sec:background:backprop} computes the gradient of the loss function, which is then used by a gradient method like \gls{sgd} to iteratively update all weights in order to minimize (or maximize) $C(\hat{y}, y)$. As we only aim to distinguish between attack and no-attack flows, and therefore have only two classes, we are using binary \gls{cel} with \textit{mean} reduction which is defined as:

\begin{equation}
C(\hat{y},y) = -\frac{1}{N}\sum_n^{N}[y_n \cdot log(\hat{y}_n) + (1-y_n) \cdot log(1-\hat{y}_n)]
\end{equation}

With $\hat{y}$ and $y$ being predicted, and target data and $N$ the batch size.

Other methods to train an \gls{ann} are e.g. the Conjugate gradient method or the Levenberg-Marquardt algorithm, but \gls{sgd} is by far the most popular.

\section{Stochastic Gradient Descent} \label{sec:background:sgd}

Stoachstic Gradient Descent is by far the most popular algorithm used for training modern neural networks. 
It is an iterative first-order optimization algorithm aimed to iteratively improve an objective function by updating its parameters towards its lowest point, i.e. the local maximum (or minimum). For non-stochastic gradient descent it must be possible to calculate an exact gradient. In the case of machine learning, the function to minimize is the sum (or mean) of the loss of all records in the dataset. The function, which is to be optimized, would then be $Q(\theta) = \sum^N_i C(g(x_i),y_i,\theta)$, where $N$ is the number of records in the dataset. Input $x_i$ and output data $y_i$ are seen as constants and $\theta$ represents all weights in the model. The length of the resulting gradient vector $\nabla_\theta Q(\theta)$ would therefore be the cardinality of theta $|\theta|$ which is the number of parameters used in the model. Function $f(\theta)$ is then a function that accumulates the loss for every record in the dataset. To put this into perspective: the datasets we use contain around 2 million records and the \gls{lstm} model contains 5 million weights. One can see that it is not possible for computers at this date to calculate an exact gradient. Therefore, a stochastic approach is used to estimate a gradient $\nabla_\theta \hat{Q}(\theta)$. The parameters of the objective function are then iteratively updated by a small amount towards the steepest slope, i.e. the gradient:

\begin{equation}
\theta = \theta - \eta\nabla_\theta Q(\theta)
\end{equation}

$\eta$ is called the \textit{learning rate} and is an important hyper parameter in machine learning, which must be tuned to the model and data at hand.

%If the set of all weight $\theta$ is described as $\theta = {w^l_{ij} | l \ell layers, i \ell nodes in, j \ell nodes out}$, 


\section{Backpropagation} \label{sec:background:backprop}

Backpropagation is a type of differentiation algorithm used to calculate the gradient of an arbitrary function with relatively low computational effort. 
During training, an input $x$ is processed and information is flowing \textit{forward} through the network, producing output $\hat{y} = g(x)$ \ref{sec:background:ann}, hence this is called a \textit{forward-pass} or \textit{forward-propagation}. The model output culminates into a single scalar cost after applying a loss function $C(\hat{y}, y)$, which can be interpreted as a measure of distance between the model output $\hat{y}$ and the target output $y$. For \gls{ml} the backpropagation algorithm is used to calculate the gradient of the loss function $\nabla_\theta C(\theta)$ with respect to every weight $w^l_{kj}$ in the model for the record that was processed. For this purpose, the weights $w$ are deemed parameters of the forward-propagation and inputs $x_i$ are deemed constant with the effect of $g(w)$ now only being dependent on $w$. The chain rule for differentiation is applied multiple times to calculate the partial derivative $\frac{\partial C(g(w),y)}{\partial w^l_{jk}}$ for every weight between every layer in the network, which ultimately yields the gradient of $C(g(w),y)$ with respect to $w$.

\section{Recurrent Neural Networks} \label{sec:background:rnn}

The broader concept behind all \glspl{rnn} is a cyclic connection which enables the \gls{rnn} to update its state based on past states and current input data \cite{rnn_review}. Typically, an \gls{rnn} consists of standard $tanh$ nodes with corresponding weights. There are different kinds of \glspl{rnn} like continuous-time and discrete-time or finite impulse and infinite impulse \glspl{rnn}. Here we will only look at discrete-time, finite impulse \glspl{rnn} as we will only be using those. This type of network, e.g. the Elman network \cite{rnn_elman}, is capable of processing sequences of variable length by compressing the information from the whole sequence into the \textit{hidden layer} or \textit{hidden state}. The model produces one output token for each input token, so the transformation is sequence2sequence where input and output sequences are of equal length. One input sequence consists of a sequence of real valued vectors $x^{(t)} = x^{(1)}, x^{(2)}, ... , x^{(T)}$ where $T$ is the sequence length. From this input sequence, an output sequence of real valued vectors $\hat{y}^{(t)} = \hat{y}^{(1)}, \hat{y}^{(2)}, ... , \hat{y}^{(T)}$ is produced. In the case of the Elman network, two parameter matrices are involved in the calculation of the output:

\begin{equation}
h^{(t)} = \sigma^h (W^h x^{(t)} + U^h h^{(t-1)} + b^h)
\end{equation}

\begin{equation}
\hat{y}^{(t)} = \sigma^{\hat{y}} (W^{\hat{y}} h^{(t)} + b^{\hat{y}})
\end{equation}

With $W^h$, $W^{\hat{y}}$ and $U^h$ being the parameter matrices, and $b^h$ and $b^{\hat{y}}$ being a parameter vectors. $\sigma^h$ and $\sigma^{\hat{y}}$ constitute (potentially different) activation functions.
To train an \gls{rnn}, pairs of input and target sequences $(x^{(t)}, y^{(t)})$ are provided, from which, analogous to the training of \glspl{ann} in general\ref{sec:background:ann}, a differentiable loss function $C(\hat{y}^{(t)}, y^{(t)})$ can be calculated which can again be minimized by applying backpropagation and \gls{sgd}. In theory, \glspl{rnn} can process data sequences of arbitrary length, but the longer the sequence, the deeper the network gets, i.e. the longer the gradient paths. This leads to complications when relevant tokens are further apart in the sequence as the \gls{rnn} is not capable of handling such "long-term dependencies" \cite{rnn_review}. Long gradient paths in \glspl{rnn} might also cause the gradient to become either very small or very large, which results in the known \textit{vanishing gradient} or \textit{exploding gradient} problems correspondingly and cause training to either stagnate or diverge. The \gls{lstm} improves upon \glspl{rnn} by making the gradient more stable and allowing long-term dependencies to be considered in the learning process.

\begin{figure}[h]
	\centering
	\includegraphics[width=1\textwidth]{img/recurrent_nn.png}
	\caption{Depiction of an unrolled \gls{rnn} with $x^{(t)}$ being the input sequence, $\hat{y}^{(t)}$ the output sequence, and $h^{(t)}$ the internal state of the \gls{rnn} after each processing stage.}
	\label{fig:background:rnn}
\end{figure}

\section{Long Short-Term Memory}

Introduced by Hochreiter and Schmidhuber in 1997 \cite{lstm_origin}, the \gls{lstm} model mitigates the vanishing and exploding gradient problem by replacing the $tanh$ nodes in the hidden layer of a conventional \gls{rnn} with \textit{memory cells} as seen in \ref{fig:background:lstm}. 
A memory cell is comprised of \textit{input node} $\tilde{C}$, \textit{hidden state} $h$, \textit{cell state} $C$, \textit{input gate} $i$, \textit{forget gate} $f$, and \textit{output gate} $o$. 

\begin{figure}[h]
	\centering
	\includegraphics[width=1\textwidth]{img/lstm_cell.png}
	\caption{One \gls{lstm} memory cell \cite{rnn_zachary}}
	\label{fig:background:lstm}
\end{figure}

In contrast to an ordinary \gls{rnn}, an \gls{lstm} has two memory states: the hidden state $h^{(t)}$ and the \textit{cell state} $C^{(t)}$. Three gates enable the cell to control the flow of information and its effects on the cell state. For this purpose, gates in an \gls{lstm} consist of a point-wise multiplication with a vector that holds values between 0 and 1. The three sigma activations seen in \ref{fig:background:lstm} produce the gate vectors. The input gate $i^{(t)}$ controls whether the memory cell is updated. The forget gate $f^{(t)}$ controls how much of the old state is to be forgotten. The output gate $o^{(t)}$ controls whether the current cell state is made visible. The weight matrices $W^i, W^j$ and $W^o$ decide how information is processed by the cell and are learned parameters. The cell state is updated by addition with the vector $\bar{C}$ after multiplication with the input gate vector $i^{(t)}$. The repeated addition of a $tanh$ activation distributes gradients and vanishing/exploding gradients are mitigated.

\begin{equation}
i^{(t)} = \sigma(W^i[h^{(t-1)},x^{(t)}] + b^i)
\end{equation}

\begin{equation}
f^{(t)} = \sigma(W^f[h^{(t-1)},x^{(t)}] + b^f)
\end{equation}

\begin{equation}
o^{(t)} = \sigma(W^o[h^{(t-1)},x^{(t)}] + b^o)
\end{equation}

\begin{equation}
\bar{C}=\tanh(W^C[h^{(t-1)},x^{t}]+b^C)
\end{equation}

\section{Attention and Transformers}

In 2017, Vaswani et al. published a paper with the ominous title "Attention is All you Need" \cite{attention_origin}, referring to the already known attention mechanism which is used to model dependencies within a data sequence over longer distances. The authors proposed the transformer model consisting entirely of self attention mechanisms to model sequences and therefore diverge from the recurrent architectures of \glspl{rnn} and \glspl{lstm}. Attention is a mechanism to capture contextual relations between tokens in a sequence, e.g. words in a sentence or packets in a flow. For every token in the input sequence, an attention vector is generated, which represents how relevant other tokens in the input sequence are to the token in question. While attention can be implemented in different ways, the authors chose the scaled dot-product attention defined as 

\begin{equation}
	Attention(Q,K,V) = softmax(\frac{QK^T}{\sqrt{d_k}})V
\end{equation}

\begin{figure}[h]
	\centering
	\includegraphics[width=0.3\textwidth]{img/attention.png}
	\caption{Self attention layer of Transformer by \cite{attention_origin}}
	\label{fig:attention}
\end{figure}

"An attention function can be described as mapping a query and a set of key-value pairs to an output" \cite{attention_origin}. $Q$, $K$ and $V$ are matrices composed of query, key and value vectors for every token with respect to every other token in the sequence.
Vaswani et al. proposed the use of Multi-Head Attention mechanism, suggesting the use of multiple independent attention heads which are generated by linear projection of the original $Q, K$ and $V$ matrices by different learned matrices $W^Q_i, W^K_i$ and $W^V_i$ for $i = 1, ... ,h$ where $h$ is the number of desired attention heads. The attention vectors of the different attention heads are again concatenated and projected by matrix $W^Z$ again resulting in a single combined attention vector instead of $h$ vectors. This results in the formulation 

\begin{equation}
	head_i = Attention(QW^Q_i, KW^K_i, VW^V_i), i = 1, ..., h
\end{equation}

\begin{equation}
	MultiHead(Q,K,V) = Concat(head_1, ..., head_h)W^O
\end{equation}

depicted in figure \ref{fig:attention}. The Multi-Head Attention block from \ref{fig:attention} is
used in the transformer encoder block \ref{fig:transformer_encoder} together with a fully-connected feed forward network. After each sub-layer (Multi-Head Attention, Feed Forward) layer, normalization is applied and a residual connection originating from the input to the sub-layer is added as can again be seen in figure \ref{fig:transformer_encoder}. The output of each sub-layer is hence defined as $LayerNorm(x + Sublayer(x))$ where $Sublayer$ is either a Feed Forward or a Multi-Head Attention function. While there is more to the transformer model, for our experiments we are only using the parts described here.

\begin{figure}[h]
	\centering
	\includegraphics[width=0.4\textwidth]{img/transformer_encoder.png}
	\caption{Transformer encoder model as proposed by \cite{attention_origin}}
	\label{fig:transformer_encoder}
\end{figure}

\section{Self-supervised Learning}

Supervised learning is most effective when teaching a \gls{nn} the desired projection, but it is limited by the amount of labeled data that is available. For many use cases, not enough is available and the cost of creating new labeled data is too high to be feasible. In those cases, self-supervised learning or self-supervised pre-training might be an efficient addition or alternative. For supervised learning, the target data provides the supervision. For self-supervised learning the data itself provides the supervision, meaning the loss $C(\hat{x},x)$ is calculated between the reconstructed input $\hat{x}$ and the actual input $x$. In general this means that some part of an input tensor or an input series is withheld and the model is tasked with reconstructing the unknown information. So instead of being trained for the task, we want it to perform, it is first trained on a \textit{proxy task} which serves no purpose on its own but forces the model to learn a semantic representation, i.e. abstract features of the data which will help solve the actual task.

\section{Auto-Encoder} \label{sec:backgrund:autoencoder}

The auto-encoder is a popular tool for self-supervised learning. The model is composed of an \textit{encoder} and a \textit{decoder} stage as can be seen in figure \ref{fig:auto_encoder}. The encoder compresses the input data, artificially causing loss of information. In the next step the decoder tries to reconstruct the compressed data as accurately as possible. The loss $C(\hat{x},x)$ is then calculated as the difference between the original input and the reconstructed one. The aim of this seemingly nonsensical task is to force the model to form an abstract, more compact representation of the input data in its restricted latent space. To compress data with minimal loss of relevant information, the network has to find patterns in the input and ideally learns some semantic or context of the data. 

\begin{figure}[h]
	\centering
	\includegraphics[width=1.0\textwidth]{img/auto_encoder.png}
	\caption{Visualization of an auto-encoder. The input is encoded and subsequently decoded yielding and approximate reconstruction of the image \cite{auto_encoders}}
	\label{fig:auto_encoder}
\end{figure}

After the self-supervised training of the auto-encoder is finished, the decoder stage is removed and subsequently the output of the encoder is used as input tensor for a classification or prediction model or the next layer of auto-encoder. 

\section{Pre-Training and Fine-Tuning}

\textit{Pre-training} with subsequent \textit{fine-tuning} describes a methodology of training a \gls{nn} in two separate phases. E.g. Google's \gls{bert} for \gls{nlp} is pre-trained in a self-supervised fashion with vast amounts of text (3.3 billion words) \cite{bert}. \textit{Self-supervised} in this context means that the input, or parts of it, are also used as the training target. Depending on the task of the model, i.e. translation, question answering, text generation, the model's parameters are then fine-tuned with labeled data to fit the given task. \textit{Fine-tuning} then involves updating the weights of a pre-trained model by training it on a task specific labeled dataset, which is usually much smaller than the dataset used for pre-training \cite{gpt3}. Up to the release date of this paper, the pre-training - fine-tuning approach is still among the most effective approaches available when it comes to training large scale (>1 billion parameters) \gls{nlp} models, but researchers have since aimed to decrease the need for labeled data even further by only presenting the model with very few, or even just one, example of a correctly executed downstream task \cite{gpt3} instead of fine-tuning on labeled data.

\section{Performance Metrics} \label{sec:background:metrics}

To measure the effectiveness of different \glspl{ids} and machine learning models in general, a commonly used set of performance metrics has been devised to promote comparison between solutions. For binary classification (attack vs. benign), the basic metrics are

\begin{itemize}
	\item \textbf{\gls{tp}}: Number of samples correctly classified as attack
	\item \textbf{\gls{tn}}: Number of samples correctly classified as benign
	\item \textbf{\gls{fp}}: Number of samples falsely classified as attack
	\item \textbf{\gls{fn}}: Number of samples falsely classified as benign
\end{itemize}

From these basic metrics, a variety of semantically more expressive metrics can be derived, which describe different performance aspects of the classification task like overall accuracy or the rate of falsely raised alarms \cite{confusion_matrix}. Commonly used metrics are 

\begin{itemize}
	\item \textbf{Accuracy} is defined as the ration of correctly classified samples to total samples. \begin{equation}
	Accuracy = \frac{TP + TN}{TP + TN + FP + FN}
	\end{equation}
	
	\item \textbf{Precision} is defined as the ration of true positive samples to predicted positive samples and represents the confidence of attack detection.
	 \begin{equation}
	Precision = \frac{TP}{TP + FP}
	\end{equation}
	
	\item \textbf{Recall or \gls{dr}} is defined as the ration of true positive samples to total positive samples. The metric describes the probability that an attack will be detected by the \gls{ids}.
	\begin{equation}
	Recall = DR = \frac{TP}{TP + FN}
	\end{equation}
	
	\item \textbf{Specificity} is defined as the ration of true negative samples to total negative samples. The metric describes the probability that a benign flow will be categorized as such by the \gls{ids}.
	\begin{equation}
	Specificity = \frac{TN}{TN + FP}
	\end{equation}
	
	\item \textbf{\gls{fnr} or \gls{mar}} is defined as the ratio of false negative samples to total positive samples and describes how many attacks go undetected by the \gls{ids}.
	\begin{equation}
	FNR = MAR = \frac{FN}{TP + FN}
	\end{equation}
	
	\item \textbf{\gls{fpr} or \gls{far}} is defined as the ratio of false positive samples to predicted positive samples and describes how often the \gls{ids} falsely raises an alarm.
	\begin{equation}
	FPR = FAR = \frac{FP}{TP + FP}
	\end{equation}
	
	\item \textbf{F1 Measure or F Score} is calculated from the precision and recall of the test and is an alternative description of the accuracy of a statistical analysis.
	\begin{equation}
	F1 = 2*\frac{Precision * Recall}{Precision + Recall}
	\end{equation}
\end{itemize}


\chapter{State of the art} \label{sec.state_of_art}

Here provide an overview of the related state of art. Look for papers that are closest to the research you are doing
Suggestion: make a table with the related papers and compare them wrt to different criteria, for instance

\begin{itemize}
	\item Findings: What do they claim (main findings)
	\item Data: What data set they are using
	\item Methods: Which methods did they use?
	\item Reproducibility: Is it possible to reproduce the results? (e.g., is the data available? are all parameter settings provided? Is source code provided?)
	\item Relevance (How relevant is it for your work)
\end{itemize}


In the last paragraph explain how your work differs from the existing works.



\newpage

%\section{Content \& Terminology} \label{sec:content}

% insert info about previous publications below
\begin{Prev.Publ}
	Parts of the contents of this chapter have been published in the following papers:
	\begin{enumerate}
		\item [\lbrack P1\rbrack] \bibentry{cen-cenelec-etsi_smart_grid_coordination_group_smart_2014} %use the \cite ciations text
		\item [\lbrack P2\rbrack] \bibentry{enisa_smart_2014}
	\end{enumerate}
	
	Explanation text, on what parts were adopted from previous publications:\\
	e.g. "The statistical anomaly detection algorithm published in the above mentioned papers and described in this
	Chapter is based on the work done in [29]."	
\end{Prev.Publ}


\subsection{Vision \tiny{My Vision on things}} \label{subsec.vision}


\subsection{Terminology    \tiny{Architecture vs Topology + Malware vs Worm}  } \label{subsec.terminology}



\newpage
%\section{Model    \tiny{WHAT will I DO? + Abstraction}  } \label{sec:model}

% insert info about previous publications below
\begin{Prev.Publ}
	Parts of the contents of this chapter have been published in the following papers:
	\begin{enumerate}
		\item [\lbrack P1\rbrack] \bibentry{cen-cenelec-etsi_smart_grid_coordination_group_smart_2014} %use the \cite ciations text
		\item [\lbrack P2\rbrack] \bibentry{enisa_smart_2014}
	\end{enumerate}
	
	Explanation text, on what parts were adopted from previous publications:\\
	e.g. "The statistical anomaly detection algorithm published in the above mentioned papers and described in this
	Chapter is based on the work done in [29]."	
\end{Prev.Publ}

\subsection{Abstraction}
- What will be abstracted?\\
- Where do I compromise?\\


\subsubsection{Architecture} \label{subsec.architecture}
add abstraction of 4 architectures here!



\subsection{Scenarios} \label{subsec.scenarios}


\subsection{Metrics    \tiny{What do i measure?}  }


\subsection{Verification / Validation     \tiny{How do I do that??}  }
Verification: “Are we building the product right”?\\
Verification is a process of evaluating the intermediary work of a software development lifecycle to check if we are in the right track of creating the final product. Checks whether the product is built as per the specified requirement and design specification.\\

Validation: “Are we building the right product”?\\
Validation is the process of evaluating the final product to check whether the software meets the business needs. In simple words the test execution which we do in our day to day life are actually the validation activity which includes smoke testing, functional testing, regression testing, systems testing etc… It determines whether the software is fit for use and satisfy the need.

\newpage
\chapter{Methodology} \label{sec:methodology}

\begin{itemize}
	\item explain why these experiments are used
	\item explain metric for comparing results (accuracy, false alarm rate)
	\item short summary of code?
\end{itemize}

Here describe the methodology you use and why you decided to use it.
e.g., theoretical considerations, simualtons, experiments, measurements, testbeds, emulations, etc. What concepts are used.

Also explain which metrics you use to measure success or failure (e.g., detection performance with accuracy, recall, precision, f1 score, RocAUC, etc.)


Provide a figure (see example figure \ref{fig:modeling-example}) to describe the processing steps

\begin{figure}[h]
	\centering
	\includegraphics[width=0.95\linewidth]{graphics/modeling-example}
	\caption{Describe in the caption exactly what can be seen in the figure}
	\label{fig:modeling-example}
\end{figure}




\newpage
\chapter{Experiments}\label{sec:experiments}

Describe all details of the experiments in a way that the reader is able to reproduce your experiments.
Provide an overview table (or multiple tables) with all experiments, data sets and all parameters used.
Give the different experiment runs a unique name and refer to this name in the results section.

Keep interim data from your experiments (e.g. predicted labels per instance, confusion matrices, etc.) in case you need to come back to the data to calculate further metrics.



\newpage

\chapter{Results} \label{sec:results}

In this chapter, we discuss results and try to explain them based on neuron activation and \gls{pd} plots. Like with the experiments section, we will be looking at the results from \gls{lstm} and the transformer model independently.

\section{Long Short-Term Memory Model} \label{sec:results:lstm}

As a baseline, we look at results where the model has been trained in a purely supervised fashion with different amounts of data of the two datasets. The results are comparable to previous experiments with deep neural networks on these datasets \cite{fog_based_detection_survey_2020} and even slightly better in some instances. Looking at tables \ref{table:results:lstm:stats_flows_supervised}, \ref{table:results:lstm:stats_flows15_supervised} we can already see that very little supervised data is needed to achieve fairly high accuracy. For the \gls{lstm} model, going from 90\% of training data (exp. 1.1.1 and 1.2.1) to 10\% (exp. 2.1.1 and 2.4.1) only amounts to an absolute drop of 0.164\% and 0.276\% accuracy for CIC-IDS2017 and UNSW-NB15 datasets respectively. Most astounding are also the results when dropping from millions of records when training with 90\% of the datasets to just the specialized subsets containing a couple of hundred entries in total and only 10 records of each attack class. Withholding most of labeled data in the datasets, this constraint only amounts to an absolute accuracy decrease of 4.114\% and 1.176\% for datasets CIC-IDS2017 and UNSW-NB15 respectively.
While this is fairly pleasant in general, it means that results will be harder to improve as any benefit pre-training provides might be overshadowed by the effectiveness of supervised training, even with very little data. \par

\input{results/results/rn500/lstm/tables/flows_supervised/stats_comparison_ALL}

\input{results/results/rn500/lstm/tables/flows_supervised/class_comparison_ALL}

\input{results/results/rn500/lstm/tables/flows15_supervised/stats_comparison_ALL}

\input{results/results/rn500/lstm/tables/flows15_supervised/class_comparison_ALL}

Next, we will be looking at results for pretraining with the different proxy tasks and different amounts of data used for supervised finetuning. Tables \ref{table:results:lstm:stats_flows_10}, \ref{table:results:lstm:stats_flows_1} and \ref{table:results:lstm:stats_flows_subset} show results for experiments 2.1.1 - 2.1.6, 2.2.1 - 2.2.6 and 2.3.1 - 2.3.6 on dataset CIC-IDS2017 conducted with 10\%, 1\% and only a very small fraction of data as defined in subset CIC17\_10. Looking at the performance metrics, we can see that there is some variance in the resulting data. The NumPy and PyTorch random seeds are the same for all experiments which means that pretraining, supervised finetuning and validation have been conducted with the exact same subsets of the original dataset which means that differences in results can only come from pretraining with different proxy tasks. This establishes the fact that pretraining in general, and also different methods of pretraining have an effect on final performance. Starting with table \ref{table:results:lstm:stats_flows_10}, we can see that pretraining with some proxy tasks improves performance while others have almost no effect or even a negative effect.  
For accuracy, the highest positive delta 0.101\% in experiments 2.1.1-6 can be observed for pretraining with the COMPOSITE proxy task \ref{sec:experiments:lstm:composite} closely followed by pretraining with the ID proxy task \ref{sec:experiments:lstm:identity} with a delta of 0.095\%. The highest negative delta in accuracy, -0.010\%, can be observed for the obscure feature proxy task \ref{sec:experiments:lstm:obscure}. It should be noted, that for detection rate, the highest delta is 0.343\% also occurring after COMPOSITE pretraining. This shows that the improvement in accuracy stems from improved attack detection capability achieved through pretraining. \par
Looking at table \ref{table:results:class_lstm:stats_flows_1} we can inspect for which attack categories improvement is most salient. When comparing training with supervised methods only in experiment 2.1.1 with the on the COMPOSITE proxy task pre-trained model in experiment 2.1.6 we can see major improvements for detection of FTP-Patator brute force attacks. Accuracy jumped from 72\% for supervised only training to 92.308\% for the COMPOSITE trained model and even 96.154\% for the PREDICT proxy task, constituting a positive delta of 24.154\%. For experiment 2.1.4, pre-training with the auto encoder task, the accuracy for detection of the FTP-Patator attack category dropped to 3.846\%. Such high variance in results shows again that the \gls{lstm} model is susceptible to different pre-training strategies. Other attack classes which have seen improvement in detection accuracy are port scans with firewall on and off (\#11-12) with positive deltas of 2.702\% and 0.606\% respectively and infiltration (\#9) and SSH-Patator (\#2) with deltas of 1.269\% and 1.185\%. \par
Looking at results for experiments 2.2.1-6 in table \ref{table:results:lstm:stats_flows_1} finetuned with 1\% of the CICIDS-2017 dataset - \textit{ceteris paribus} - the maximum delta in accuracy increased to 0.178\%, which in this iteration is observed after pretraining with the PREDICT proxy task \ref{sec:experiments:lstm:predict_packet}. The positive delta for the COMPOSITE proxy tasks increased to 0.136\% and the delta for the ID proxy task remained almost the same at 0.094\%. PREDICT, ID and COMPOSITE proxy tasks have shown improvements in accuracy and performance overall for experiments 2.1.1-6 and 2.2.1-6, but the pattern breaks down when looking at experiments 2.3.1-6 with finetuning performed with subset CIC17\_10 where improvement is now only present for pretraining with the ID proxy task where the positive deltas in accuracy and detection rate have increased to 0.594\% and 3.602\% respectively. All other pretraining resulted in strongly reduced performance most salient with the OBSCURE proxy task with a negative delta in accuracy of -2.327\%. \par 
It shall be noted that training with this little data entails a high variability in validation accuracy and loss over the course of training. In our case, validation accuracy was tested only every 6 epochs of training for the overall 600 training epochs to keep training times reasonable. Higher accuracy scores might have occurred during
epochs in which validation accuracy was not tested before the model started overfitting. The same caveat must be stated for experiments 2.2.1-6 or 2.5.1-6 where validation accuracy was tested only every second epoch during training, but training for these experiments was much more stable in general and 1/2 is a much higher chance of catching the highest result than 1/6. It also shall be noted that for training with the CIC17\_10 and UNSW15\_10 the ratio of samples between categories has changed when compared to the original dataset or the stratified sampled 10\% and 1\% subsets. For CIC17\_10 and UNSW15\_10, the same amount of samples per category are included. This does not impact the comparison between pre-trained and non-pre-trained models but takes from the comparability between results of experiments 2.3.1-6 and the results of experiments 2.1.1-6 and 2.2.1-6. \par
A similar pattern emerges from the results of tests on the UNSW-NB15 dataset. For experiments 2.4.1-6 in table \ref{table:results:lstm:stats_flows15_1} finetuned with 10\% of the dataset minor improvement can again be observed for the pre-trained models. The highest positive delta of 0.086\% occurred after pre-training with de identity function proxy task in experiment 2.4.5 when comparing to the purely supvervised training in experiment 2.4.1. Training with other proxy tasks shows comparably minor improvements to accuracy. The maximum delta between supervised only trained and pre-trained models increases to 0.137\% in experiments 2.5.1-6, this time occurring for proxy task AUTO with all other proxy tasks also showing minor improvements in accuracy. The pattern breaks again when looking at finetuning with the specialized subset UNSW15\_10 in table \ref{table:results:lstm:stats_flows15_subset} for experiments 2.6.1-6. Here, almost no improvement is measurable and for the COMPOSITE proxy task accuracy even dropped by 0.168\% when for finetuning with 1\% and 10\% it showed slight improvement. The only increase in accuracy is measurable for the AUTO proxy task with an accuracy increase of 0.01\% might well be just noise.

\input{results/results/rn500/lstm/tables/flows_10/stats_comparison_ALL}

\input{results/results/rn500/lstm/tables/flows_10/class_comparison_ALL}

\input{results/results/rn500/lstm/tables/flows_1/stats_comparison_ALL}

\input{results/results/rn500/lstm/tables/flows_1/class_comparison_ALL}

\input{results/results/rn500/lstm/tables/flows_subset/stats_comparison_ALL}

\input{results/results/rn500/lstm/tables/flows_subset/class_comparison_ALL}

\input{results/results/rn500/lstm/tables/flows15_10/stats_comparison_ALL}

\input{results/results/rn500/lstm/tables/flows15_10/class_comparison_ALL}

\input{results/results/rn500/lstm/tables/flows15_1/stats_comparison_ALL}

\input{results/results/rn500/lstm/tables/flows15_1/class_comparison_ALL}

\input{results/results/rn500/lstm/tables/flows15_subset/stats_comparison_ALL}

\input{results/results/rn500/lstm/tables/flows15_subset/class_comparison_ALL}

\section{Transformer Model} \label{sec:results:transformer}

\input{results/results/rn500/transformer/tables/flows_supervised/stats_comparison_ALL}

\input{results/results/rn500/transformer/tables/flows_supervised/class_comparison_ALL}

\input{results/results/rn500/transformer/tables/flows15_supervised/stats_comparison_ALL}

\input{results/results/rn500/transformer/tables/flows15_supervised/class_comparison_ALL}

\input{results/results/rn500/transformer/tables/flows_10/stats_comparison_ALL}

\input{results/results/rn500/transformer/tables/flows_10/class_comparison_ALL}

\input{results/results/rn500/transformer/tables/flows_1/stats_comparison_ALL}

\input{results/results/rn500/transformer/tables/flows_1/class_comparison_ALL}

\input{results/results/rn500/transformer/tables/flows_subset/stats_comparison_ALL}

\input{results/results/rn500/transformer/tables/flows_subset/class_comparison_ALL}

\input{results/results/rn500/transformer/tables/flows15_10/stats_comparison_ALL}

\input{results/results/rn500/transformer/tables/flows15_10/class_comparison_ALL}

\input{results/results/rn500/transformer/tables/flows15_1/stats_comparison_ALL}

\input{results/results/rn500/transformer/tables/flows15_1/class_comparison_ALL}

\input{results/results/rn500/transformer/tables/flows15_subset/stats_comparison_ALL}

\input{results/results/rn500/transformer/tables/flows15_subset/class_comparison_ALL}

\begin{itemize}
	\item maximum accuracy with 0-90-10 pre-sup-val training
	\item comparison between pretraining accuracy with different proxy tasks for 10-80-10 pre-sup-val training
	\item comparison between pretraining accuracy with different proxy tasks for 1-89-10 pre-sup-val training
	\item comparison between pretraining accuracy with different proxy tasks for subset 10\_flows subset pre-sup-val training
	\item comparison of performance improvements for different amounts of supervised training
	\item comparison of performance improvements for different compositions of pretraining data
	\item comparison between multiple datasets
	\item comparison to orthogonal initialization/random initialization/0-initialization
	\item comparison of validation loss and accuracy convergence for different pretraining tasks
	\item provide data for maximum results including class stats for both datasets to establish a feel for the maximally possible accuracy with supervised training and 90\% of data
	\item show results for different amounts of supervised data and discuss results between different proxy tasks by showing loss progression and validation accuracy over training time and comparing class stats
	\item highlight the improvement in accuracy when comparing to supervised training only
	\item look closely at differences in loss progression and validation accuracy over time between different proxy tasks
\end{itemize}

\section{Explainability}



\begin{itemize}
	\item close look at differences in performance for different attack classes
	\item partial dependency plots
	\item neuron activation
	\item \gls{dlstm}s and transformer encoder already very effective, so improvement is hard
\end{itemize}

\newpage

\chapter{Discussion} \label{sec:discussion}

In this thesis we set out to inspect the possible benefits of pre-training on \gls{lstm} and transformer encoder models to answer the questions:

\begin{itemize}
	\item R1: Can self-supervised pre-training improve the flow classification capabilities of an \gls{lstm} model?
	\item R2: Can self-supervised pre-training improve the flow classification capabilities of a transformer encoder model?
	\item R3: Which pre-training tasks improve accuracy and which do not?
	\item R4: If improvement is possible, how can it be explained?
\end{itemize}

The results of our experiments appear to be inconclusive. Some experiments have shown minor improvements, but later experiments with the same proxy task and less data have shown either no improvement or worse performance. As can be seen in tables \ref{table:discussion:lstm:accuracy_differences} and \ref{table:discussion:transformer:accuracy_differences}, the \gls{lstm} model seems to be more receptive for pre-training but no clear pattern emerges which could point at a proxy-task that generally improves accuracy when used in pre-training. The first and second research questions, R1 and R2, must therefore be answered with "it depends":

\begin{itemize}
	\item A1: Yes, pre-training can improve flow classification accuracy of \gls{lstm} models in some instances, but it can not be concluded that this is generally so.
\end{itemize}

Referring to section \ref{table:discussion:transformer:accuracy_differences} it can be seen that pre-training seems to have even less positive impact on the transformer model compared to the \gls{lstm} model in general, but still yielded some positive results for experiments with very little supervised data i.e. the dedicated subsets used for experiments 3.3.3 and 3.6.4 as can be seen in tables \ref{table:results:transformer:stats_flows_subset} and \ref{table:results:transformer:stats_flows15_subset}.

\begin{itemize}
	\item A2: Yes, pre-training can improve flow classification accuracy of transformer encoder models in very specific instances, but in most cases, it does not.
\end{itemize}

We could name the predict packet (PREDICT) and surprisingly the identity function (ID) proxy task as most likely to improve final accuracy of the \gls{lstm} model. In table \ref{table:discussion:lstm:accuracy_differences} we can see that the highest overall absolute increase in accuracy of 0.594\% was also achieved with pre-training with the ID proxy task in experiment 2.3.5 \ref{table:results:lstm:stats_flows_subset}. \par

\begin{itemize}
	\item A3: The \textit{predict packet (PREDICT)} and \textit{identity function (ID)} proxy tasks where most effective in increasing flow classification accuracy for \gls{lstm} models in our experiments. For transformer encoder models it was the \textit{auto encoder (AUTO)} proxy task with 20\% dropout rate.
\end{itemize}

As is often the case, complete explainability for \glspl{nn} is hard to achieve. From the \gls{pd} plots we can deduce that pre-training had a significant impact on the internal behavior of the models. The differences in neuron activations after pre-training with different proxy tasks suggests the same thing. We can therefore conclude that the internal representation of data is influenced in a positive way in the instances where pre-training improved classification accuracy when compared to a model with randomly initialized weights.

\begin{itemize}
	\item A4: Pre-training can be seen as a non-random weight initialization before conducting supervised training which leads to an alternate internal data representation in the hidden state and ultimately to an increase in flow classification accuracy.
\end{itemize}

The fact that any improvement was observable at all is already promising, when considering that the baseline results of exclusively supervised trained models are already very high when compared to other state-of-the-art results for the used datasets in contemporary research. Increasing accuracy ever so slightly without the need for more labeled data might make an \gls{ml} based \gls{ids} feasible in a real world scenario when before it was not. There could be several reasons why no conclusive pattern can be found in the results: \par

\begin{table}[]
	\centering
	\scalebox{0.9}{
		\begin{tabular}{rccccc}
			\thead{Experiments (\#)} & \thead{PREDICT(2)} & \thead{OBSCURE(3)} & \thead{AUTO(4)}  & \thead{ID(5)}            & \thead{COMPOSITE(6)} \\ \midrule
			\multicolumn{6}{c}{CIC-IDS2017} \\ \midrule
			10\% (2.1.x)        & 0.073\%    & -0.012\%   & -0.002\% & 0.095\%          & \textbf{0.101\%}      \\
			1\% (2.2.x)         & \textbf{0.178\%}    & -0.075\%   & -0.007\% & 0.094\%          & 0.136\%      \\
			CIC2017\_10 (2.3.x) & -1.609\%   & -2.327\%   & -0.744\% & \textbf{0.594\%} & -1.453\%     \\ \midrule
			\multicolumn{6}{c}{UNSW-NB15} \\ \midrule
			10\% (2.4.x)        & 0.083\%    & 0.067\%    & 0.045\%  & \textbf{0.086\%}          & 0.037\%      \\
			1\% (2.5.x)         & 0.108\%    & 0.043\%    & \textbf{0.137\%}  & 0.099\%          & 0.024\%      \\
			UNSW15\_10 (2.6.x)  & 0.005\%    & -0.115\%   & \textbf{0.010\%}  & -0.089\%         & -0.168\%     \\ \midrule
			\makecell{Max. absolute \\ accuracy increase} & 0.178\%    & 0.067\%    & 0.137\%  & \textbf{0.594\%}          & 0.136\%      \\
		\end{tabular}
	}
	\caption{Absolute differences of validation accuracy between differently pre-trained \gls{lstm} model and the same model without pre-training. The highest value of each row is marked bold.}
	\label{table:discussion:lstm:accuracy_differences}
\end{table}

\begin{table}[]
	\centering
	\resizebox{0.7\linewidth}{!}{\begin{tabular}{rlll}
		\thead{Experiments (\#)} & \thead{MASK(2)} & \thead{OBSCURE(3)} & \thead{AUTO(4)}  \\ \midrule
		\multicolumn{4}{c}{CIC-IDS2017} 											 \\ \midrule
		10\% (3.1.x)        & \textbf{-0.037\%} & -0.724\%          & -0.135\%         \\
		1\% (3.2.x)         & -0.084\%         & \textbf{-0.021\%}         & -0.098\%         \\
		CIC2017\_10 (3.3.x) & -1.055\%         & \textbf{0.678\%} 	& -0.756\%         \\ \midrule
		\multicolumn{4}{c}{UNSW-NB15} 												 \\ \midrule
		10\% (3.4.x)        & -1.817\%         & -0.361\%         & \textbf{-0.018\%}         \\
		1\% (3.5.x)         & -0.163\%         & -0.072\%         & \textbf{0.003\%}          \\
		UNSW15\_10 (3.6.x)  & -1.213\%         & -0.228\%         & \textbf{0.295\%} \\ \midrule
		\makecell{Max. absolute \\ accuracy increase}    & 0.037\%          & \textbf{0.678\%} & 0.295\%          \\
	\end{tabular}}
	\caption{Absolute differences of validation accuracy between differently pre-trained transformer model and the same model without pre-training. The highest value of each row is marked bold.}
	\label{table:discussion:transformer:accuracy_differences}
\end{table}


The achieved accuracy scores by the models when fine-tuned with 90\% of data were in the high 99.x percent \ref{table:results:explainability:model_comparison} and even with only one percent of data used for training, accuracy was still over 99\%. The flow representation used omits the packet payload, so payload based attacks like \textit{SQL Injection} or \textit{XSS} attacks are very hard, if not impossible, to detect. There might simply be very little room for improvement. So even if pre-training were effective, it would not show up in our results. One approach to mitigate this effect would be to switch to multinomial classification of the exact attack type instead of binary classification. This approach increases fine-tuning training time, but since 
unsupervised pre-training takes disproportionately longer than fine-tuning at the moment, overall training time would not increase significantly. \par

In accordance with the last point, the datasets we used seem to be easy to classify. Even though not completely comparable, the results from the \gls{dtc} alone show that records in both datasets can be classified with high accuracy, i.e. about 97\% or more \ref{table:results:explainability:model_comparison} even without access to flow information.
The lack of complexity of the data might be another reason why pre-training in this case showed no improvement as supervised training was always sufficient to extract all relevant information from the input data. To inspect this conjecture, a more challenging dataset should be used or a combination of different datasets. Ideal would be actual captured traffic of a mid size company during a penetration test. \par

Word embeddings are an integral part of modern \gls{nlp} systems. This aspect is not included in our process since our feature vectors were already given. As our feature representation contains some qualitative features which are mapped to a mostly arbitrary number e.g. the IP protocol identifier and port numbers, creating embeddings for qualitative features and tuning them to the task at hand might lead to improvement in overall accuracy, independent of whether pre-training is used or not. \par
	
Insufficient data or model complexity might be another reason why pre-training has produced no consistent improvement. Google BERT \cite{bert} uses a model with 110 million parameters in its \textit{BASE} version and 340 million in for its \textit{LARGE} version. In comparison, our \gls{lstm} model consists of 5.3 million parameters and the transformer encoder model only contains 74 thousand parameters. In the original paper about BERT, google claims to have pre-trained their model on a corpus of about 3.3 billion words. During our work on this thesis many new \gls{nlp} models like Googles T5 \cite{google_t5}, Nvidias Megatron \cite{megatron} were presented with the largest one being OpenAIs GPT-3 with 175 billion parameters, pre-trained on 410 billion tokens \cite{gpt3}. The datasets we are using contain around 25 million packets each which is two magnitudes smaller than the corpus Googles BERT was trained on. Larger models or significantly larger datasets would not have been feasible in our case due to lack of computational resources. One of these two reasons, i.e. either lack of model complexity or pre-training data, might be the cause of our inconclusive results. This hypotheses is of course not easily checked without the necessary resources. As already stated above, the dataset used for unsupervised pre-training could be magnified by merging multiple datasets together or even using available unlabeled network traffic data for pre-training. \par

Unsuited proxy tasks could also be a reason why pre-training showed little effect. The used proxy tasks appeared to us as most intuitive but other choices might have been more effective. There is also the possibility of pre-training using labeled data where e.g. a custom dataset is constructed where flows are labeled with the application layer protocol the flow is part of. Other unsupervised training methods might also be feasible e.g. energy-based unsupervised learning \cite{energy_based_training}. The possible approaches here are many and we have by no means explored all our options. \par

Looking at the validation and training loss progression during supervised fine-tuning, it shows that especially in experiments where models are trained on only 1\% of the dataset or even less, i.e. with the specialized subsets, the models start to overfit heavily. It might be that the effects of over-fitting in these scenarios mitigates any positive effect pre-training might have had. During fine-tuning the model learns two things: Patterns in the data based on the label, but also how to classify data at all. At the very least, the fully connected layer must be trained for classification in the fine-tuning phase. During pre-training the model might have constructed a perfectly useful latent feature space, but it has not yet learned how to do actual classification based on it. This poses the difficulty of teaching the model how to classify, but not overfit it on the little data it has, overwriting any possible gains from pre-training. Deducted from this hypothesis, it should be a requirement that the time the model takes to converge should be greater than the time the model learns to classify at all. This problem is implicitly mitigated by increasing the size of the dataset.\par

Furthermore, pre-training was performed with parts of the same dataset which was also used for supervised fine-tuning and for validation. Even though the labels where ignored, the data used for pre-training contained attack flows. If anything, this is most likely beneficial for possible positive effects on the final accuracy and metrics and has to be considered when trying to reproduce the results. This effect could be mitigated by e.g. using one dataset for pre-training and a different dataset for fine-tuning. In a real world setting however our scenario is more likely to apply as for pre-training a \gls{nids} it would make most sense to train the model with network traffic captured within the network the system is going to protect. This would also mean that it has to be assumed that the data also contains some attack flows as there is no way of asserting otherwise. For a real world application this approach also assures that the models learn common patterns of the specific networks they are going to be used in. It might even make sense to re-train or at least update the model periodically with recent data. Unfortunately network protocols are not as universal as natural language which makes transfer learning more difficult .i.e. it would be hard to teach the model universal patterns of network traffic. An ad-hoc approach tailored to the traffic of a single specific network is more likely to yield usable results in the near future. \par

An unsuited data representation might stifle the effectiveness of the used models. Even after deciding to use a flows representation, there is a wide selection of feature spaces \cite{feature_vectors}.
Our analysis of the datasets in section \ref{sec:results:explainability:dtc}, in particular table \ref{table:results:expl:fimp_90_ds}, shows that a lot of importance is accumulated in a few features while others are mostly irrelevant which shows that the selection of the correct feature and data representation has a significant effect on the performance of the models. Omitting unused or irrelevant features from the data representation would drastically decrease training time, but it is of course impossible to tell \textit{a priori} if a feature is going to be relevant for classification of the data at hand. There are however feature selection and feature reduction methods that can and should be applied. \par
We used a flow representation of per-packet feature vectors instead of the often used approach of aggregating all packets of a flow into a single feature vector containing statistical data. This was, as already explained in earlier sections, to enable the use of machine learning techniques used in \gls{nlp} which almost exclusively expect a sequences of tokens as input. When considering the immense overhead of using sequences instead of a single vector and looking at the results of the \gls{dtc} which has no concept of flows but still performs reasonably well, there might be better ways to build meaningful sequences. A possible approach is to use a sequence constituting of aggregated data over specified consecutive time frames comes to mind, which has already been used in other papers like \cite{attention_model_ids}, \cite{kitsune}. \par

\begin{table}[]
	\centering
	\begin{tabular}{rccccc}
		\thead{Experiment \#} & \thead{PREDICT(2)} & \thead{OBSCURE(3)} & \thead{AUTO(4)}   & \thead{ID(5)}      & \thead{COMPOSITE(6)} \\ \midrule
		\multicolumn{6}{c}{CIC-IDS2017} \\ \midrule
		10\% (2.1.x) 		& Yes    & No   & No   & Yes    & Yes      \\ 
		1\% (2.2.x) 		& Yes    & No   & No   & Yes    & Yes      \\ 
		CIC2017\_10 (2.3.x) 	& No   	 & No   & No   & Yes    & No     \\ \\ \midrule
		\multicolumn{6}{c}{UNSW-NB15} \\ \midrule
		10\% (2.4.x) 		& Yes    & Yes    & Yes    & Yes    & Yes      \\ 
		1\% (2.5.x) 		& Yes    & Yes    & Yes    & Yes    & Yes      \\ 
		UNSW15\_10 (2.6.x) 		& Yes    & No   & Yes    & No   & No     \\ \midrule
		\makecell{\# Cases in which \\ pre-training \\ improved accuracy} & 5/6 & 2/6 & 3/6 & 5/6 & 4/6  
	\end{tabular}
	\caption{Table of comparisons whether accuracy improved for pre-trained \gls{lstm} models when compared to supervised only trained baseline experiments.}
	\label{table:discussion:lstm:improvement_results}
\end{table}

\begin{table}[]
	\centering
	\begin{tabular}{rccc}
		\thead{Experiments (\#)} & \thead{MASK(2)} & \thead{OBSCURE(3)} & \thead{AUTO(4)} \\ \midrule
		\multicolumn{4}{c}{CIC-IDS2017} \\ \midrule
		10\% (3.1.x)   & No  & No   & No   \\
		1\% (3.2.x)    & No  & No   & No   \\
		CIC2017\_10 (3.3.x) & No  & Yes  & No   \\ \midrule
		\multicolumn{4}{c}{UNSW-NB15} \\ \midrule
		10\% (3.4.x)     & No  & No   & No   \\
		1\% (3.5.x)      & No  & No   & Yes  \\
		UNSW15\_10 (3.6.x)   & No  & No   & Yes  \\ \midrule
		\makecell{\# Cases in which \\ pre-training \\ improved accuracy} & 0/6 & 1/6 & 2/6
	\end{tabular}
	\caption{Table of comparisons whether accuracy improved for pre-trained transformer models when compared to supervised only trained baseline experiments.}
	\label{table:discussion:transformer:improvement_results}
\end{table}

While machine learning researchers in the realm of \gls{nlp} are already trying to move past the pre-training / fine-tuning paradigm \cite{gpt3}, it is yet to be explored in the context of \gls{nid}. We contributed by applying these established methods in an attempt to harness them for the very needed improvement of
machine learning based \gls{nid}. Although by no means exhaustive, we managed to achieve a useful primer by using both state-of-the-art models and techniques combined with careful inspection of the obtained results and the used data. The tables \ref{table:discussion:lstm:improvement_results} and \ref{table:discussion:transformer:improvement_results} serve as a final overview on the instances in which pre-training actually improved classification accuracy. Considering the difficulties and shortcomings stated above, the fact that it was possible to improve classification accuracy in some instances remains an encouraging pointer towards the feasibility of the approach overall. Taking into account that the datasets used seem to be a mismatch for our experiments and the fact that results from experiments with the specialized subsets should be ignored due to the negative effects of over-fitting, the results for the \gls{lstm} model become a lot more promising. 
The facts that improvements where only minor and that our experiments where quite narrow in the sense that we only performed them on two datasets and with the same random seed, remain. For these reasons further inquiry is needed, applying the lessons we learned during our research, to confirm that the results are generalizable for sequence2sequence models in the context of deep learning based \gls{nid}. \par

\newpage
\chapter{Conclusion} \label{sec:conclusion}

Conclude your work. Stress again what was the contribution. 
Provide an outlook what could be further improvements and what could future research do to continue your work.

\newpage
%\section{References}
\renewcommand\refname{\vskip -1cm}   %hide the auto generated "Reference" header creaded by \bibliography{bibliography}. As it does not contain a chapter number. 
\bibliographystyle{plain}
\bibliography{bibliography} 
\newpage


\appendix
%\section{First Appendix}
%\chapter {Chapter Appendix}
\appendix

\section{Appendix}
TEXT

%\section*{Table of Abbreviations}\label{cha_Abkurzungen}
%
%\begin{acronym} [emem]  %[SEPSEP] 
%	\acro{DDoS}[DDoS]{Distributed Denial of Service}
%	\acro{HAN}[HAN]{Home Area Network}
%	\acro{ICT}[ICT]{Information- and Communication Technology}
%	\acro{NAN}[NAN]{Neighborhood Area Network}
%	\acro{PTP}[PTP]{Precision Time Protocol}
%	\acro{QoS}[QoS]{Quality of Service}
%	\acro{SCADA}[SCADA]{System Control and Data Acquisition}
%	\acro{TU Wien}[TU Wien]{Vienna University of Technology}
%\end{acronym}
%
%\section*{Mathematical Notation}\label{cha_notation}
%
%\listoffigures
%
%\listoftables


% Remove following line for the final thesis.
%\input{chapters/latex-intro.tex} % A short introduction to LaTeX.

\backmatter

% Use an optional list of figures.
\listoffigures % Starred version, i.e., \listoffigures*, removes the toc entry.

% Use an optional list of tables.
\cleardoublepage % Start list of tables on the next empty right hand page.
\listoftables % Starred version, i.e., \listoftables*, removes the toc entry.

% Use an optional list of alogrithms.
%\listofalgorithms
\addcontentsline{toc}{chapter}{List of Algorithms}

% Add an index.
\printindex

% Add a glossary.
\printglossaries

%\appendix
%\section{First Appendix}
%\chapter {Chapter Appendix}

% Add a bibliography.
\bibliographystyle{alpha}
\bibliography{intro}

\end{document}